\documentclass[handout]{beamer}
%\documentclass{beamer}

\usepackage[utf8]{inputenc}
\usepackage{default}

\usepackage{graphicx}
\usepackage{verbatim}
\usepackage{pgfpages}
\usepackage{epstopdf} % to have eps loaded
\usepackage[normalem]{ulem} % to have strike-through
\usepackage{alltt} % verbatim with commands inside
\usepackage{hyperref}

% \usepackage[latin1]{inputenc}
\usetheme{Warsaw}

% New commands
\newcommand{\cfgopt}[1]{{\it-*#1 config option}}
\newcommand{\signal}[1]{{\it#1}}
\newcommand{\feedback}[1]{{\tiny (#1 feedback)}}
\newcommand{\code}[1]{{\tiny #1)}}
\newcommand{\regbit}[1]{({\bf#1})}
\newcommand{\tclpackage}[1]{{\alert{#1}}}
\newcommand{\octopus}{\includegraphics[height=2ex]{../logo/octobiwan.png}}

\title[Octopus]{Octopus}
\subtitle{TCL Packages for Your Daily Work}

\author{Octavian Petre}
\institute{}
\date{April 17, 2013}
\titlegraphic{\includegraphics[height=.5\textheight]{../logo/octobiwan.png}}

\logo{\includegraphics[height=15px]{support_pictures/export/octobiwan_icon.jpg}}
% to have page numbers
% This is copied from /usr/share/texmf-site/tex/latex/beamer/base/themes/outer/beamerouterthemeinfolines.sty
% which is used in Madrid theme
\setbeamertemplate{footline}
{%
  \leavevmode%
  \hbox{%
  \begin{beamercolorbox}[wd=.333333\paperwidth,ht=2.25ex,dp=1ex,center]{author in head/foot}%
    \usebeamerfont{author in head/foot}\insertshortauthor(\insertshortinstitute)
  \end{beamercolorbox}%
  \begin{beamercolorbox}[wd=.333333\paperwidth,ht=2.25ex,dp=1ex,center]{title in head/foot}%
    \usebeamerfont{title in head/foot}\insertshorttitle
  \end{beamercolorbox}%
  \begin{beamercolorbox}[wd=.333333\paperwidth,ht=2.25ex,dp=1ex,right]{date in head/foot}%
    \usebeamerfont{date in head/foot}\insertshortdate{}\hspace*{2em}
    \insertframenumber{} / \inserttotalframenumber\hspace*{2ex} 
  \end{beamercolorbox}}%
  \vskip0pt%
}

\begin{document}
\setcounter{tocdepth}{1}

\begin{frame}[plain]
\titlepage 
\tiny{Get the latest presentation from \href{https://github.com/octavsly/octopus}{\underline{Octopus Repository}}}
\end{frame}

\AtBeginSection[]
{
	  \begin{frame}<beamer>
		    \frametitle{Outline}
		    \tableofcontents[currentsection,currentsubsection]
	  \end{frame}
}


\begin{frame}
	\frametitle{Outline}
	\tableofcontents[pausesections]
\end{frame}

\section{What is Octopus(\octopus)?}
\begin{frame}{What is Octopus(\octopus)?}
	\begin{itemize}[<+->]
	\item Collection of TCL packages that should help the development of TCL scripts
		\begin{itemize}
		 \item A TCL package is just a collection of useful procedures.
		\end{itemize}
	\item Available \octopus packages 
		\begin{itemize}
		 \item \tclpackage{octopus} \\TCL package.
		 \item \tclpackage{octopusRC} \\RTL Compiler package. Depending on \tclpackage{octopus}.
		 \item \tclpackage{octopusDS} \\DesignSync package. Depending on \tclpackage{octopus}.
		 \item \tclpackage{octopusNC} \\NCsim package. Depending on \tclpackage{octopus}.
		\end{itemize}
	\item Other packages will be developed as needed.
	\end{itemize}
\end{frame}

\section{WHY?/HOW?}
\subsection{WHY?}
\begin{frame}{WHY?}
	\begin{itemize}[<+->]
	 \item I am unable to easily parse many nested procedures for which arguments are parsed by position. No clear API.\\
	 Thus, \alert{code readability}.
	 \item Inability to enhance procedures without breaking all calls, or use tricks such as: default values or {\bf args} variable. \\
	 Thus, \alert{portability}.
	 \item Lack of documentation of procedures (RC scripts, test benches, ncsim.... etc.)\\
	 Thus, \alert{documentation}.
	 \item Structural reuse as opposed to adhoc.\\
	 Thus, \alert{traceability}.
	 
	\end{itemize}
\end{frame}

\subsection{HOW?}
\begin{frame}[fragile]{HOW?}{1}
	\begin{itemize}
	\item \alert{code readability} \& \alert{portability} \\
	\item<2->{Instead of positional arguments\\}
	\end{itemize}
	\begin{overprint}
	\onslide<2|handout:0>
		\begin{semiverbatim}
		set\_attribute\_recursive dont\_touch true [find * ... ] up
		\end{semiverbatim}
	\onslide<3-|handout:1>
		\begin{semiverbatim}
		\sout{set_attribute_recursive dont_touch true [find * ... ] up}
		\end{semiverbatim}
	\end{overprint}
	
	\pause
	\begin{itemize}
	\item<3> {Use options for calling procedures/scripts:}
	\end{itemize}
	\begin{semiverbatim}
	\uncover<3>{::octopusRC::set_attribute_recursive \textbackslash
{    }--attribute dont_touch true \textbackslash
{    }--objects [find * ... ] \textbackslash
{    }--direction up}
	\end{semiverbatim}
\end{frame}

\begin{frame}[fragile]{HOW?}{2}
	\begin{itemize}
	\item \alert{portability} \\
	Easy to extend procedure without breaking previous calls\\
	\begin{semiverbatim}
	::octopusRC::set_attribute_recursive \\
	    --attribute dont_touch true \\
	    --objects [find * ... ] \\
	    --direction up \\
	    \alert{--ignore_clocks} # THIS IS A NEW OPTION!!!
	\end{semiverbatim}
	\end{itemize}
\end{frame}
\begin{frame}[fragile]{HOW?}{3}
	\begin{itemize}
	\item \alert{documentation}\\
	Built in. Serves as comments as well.\\
	\end{itemize}
	{\tiny
	\begin{semiverbatim}
set var_array(10,attribute) ..."\alert{Specify the attribute to be applied. Format is: attribute <true|false>.}"]
set var_array(30,objects)   ..."\alert{Specify the objects for which the attributes will beapplied. e.g. instaces/modules/pins/etc.}"]
set var_array(40,direction) ..."\alert{Specify the direction of recursion. up: all parents will get the attribute, down: all children. both:}"]

set help_tail \{
        puts "\alert{More information:}" 
        puts "\alert{    --objects:   While pins, or other objects, can be specified, it makes no sense, since recursion is not yet implemented}"
\}
	\end{semiverbatim}
	}
\end{frame}

\section{Quick Start/Installation}
\subsection{Quick Start}
\begin{frame}[fragile]{Quick Start}
	\begin{itemize}
	\item Add OCTOPUS\_INSTALL\_PATH variable to the shell environment. For bash:
	\end{itemize}
	{\small
	\begin{semiverbatim}
\alert{export OCTOPUS_INSTALL_PATH=<directory location of octopus.tcl>}
	\end{semiverbatim}
	}
	\begin{itemize}
	\item Add to your TCL script:
	\end{itemize}
	{\small
	\begin{semiverbatim}
\alert{lappend auto_path \$env(OCTOPUS_INSTALL_PATH)}
\alert{package require octopus   0.1}
# using procedures without ::octopus:: prefix\footnote[1]{\tiny{Don't use it for OctopusRC. Conflicts with RC procedures.}}
namespace import ::octopus::*
	\end{semiverbatim}
	}
\end{frame}

\subsection{Installation}
\begin{frame}[fragile]{Installation}
	\begin{itemize}
	\item<1-> Latest version available via git repository \\
	\begin{semiverbatim}
	\alert{git clone https://github.com/octavsly/octopus.git}
	\end{semiverbatim}
	\item<2-> Subsequent updates 
	\begin{semiverbatim}
	\alert{git pull}
	\end{semiverbatim}
	\end{itemize}
\end{frame}

\section{Documentation}
\begin{frame}[fragile]{Documentation}
	\begin{itemize}
	\item \href{https://github.com/octavsly/octopus/wiki}{Octopus Wiki Page} available \\
		Wiki behind the development scripts.
	\pause
	\item Help available from every procedure using \alert{-{}-help} option.
	\pause
	\item Standalone utility to display help of all/any/many procedures:
	\pause
	\end{itemize}
	{\small
	\begin{semiverbatim}
\alert{joe@moon> export OCTOPUS_INSTALL_PATH=<location of octopus.tcl>}
\alert{joe@moon> ./info_procedures.tcl}
	\end{semiverbatim}
	}
\end{frame}

\section{Under the Hood}
\subsection{Octopus}
\begin{frame}[fragile]{Octopus}
	\begin{itemize}
	 \item Procedures/Scripts argument parsing.
	 \item Messages handling.
	 \item Flow Control.
	 \item Colours, etc.
	\end{itemize}

	{\small
	\begin{verbatim}
           ::octopus::extract_check_options_data
           ::octopus::display_message
           ::octopus::summary_of_messages
        
           ::octopus::set_octopus_color
           ::octopus::abort_on
           ::octopus::parse_file_set
           ::octopus::debug_variables
	\end{verbatim}
	}
\end{frame}

\subsection{::octopus::extract\_check\_options\_data}
\begin{frame}[fragile]{::octopus::extract\_check\_options\_data}{1}
	\alert{The Core Procedure}. \\Can parse procedures and/or script arguments.

	{\tiny
	\begin{semiverbatim}
proc ::octopusRC::define_dft_test_signals \alert{args} \{
    set help_head \{
        ::octopus::display_message none "Extracts DfT constraints from SDC set_case_analysis statements"
    \}
    # Procedure options parsing
    set \alert{var_array}(10,timing-modes)	[list "--timing-modes" "{<none>}" "string" "1" "infinity" "\$timing_modes" \textbackslash
        "The timing mode(s) the set_case_analysis will be extracted from" ]
      ...
    \alert{extract_check_options_data}
	\end{semiverbatim}
	}
\end{frame}

\begin{frame}[fragile]{::octopus::extract\_check\_options\_data}{2}
	{\tiny
	\begin{semiverbatim}
	set var_array(10,timing-modes)	[list "\alert<1>{--timing-modes}" "\alert<2>{<none>}" "\alert<3>{string}" "1" "infinity" "T1 T2" \
		"The timing mode(s) the set_case_analysis will be extracted from" ]
	\end{semiverbatim}
	}
	\begin{itemize}[<+->]
	\item [{\tiny -{}-timing-modes}] Command line/Procedure \alert{option}. \\
	$\left\langle orphaned \right\rangle$ can be used for arguments without command line option.
	\item [$\left\langle none \right\rangle$] Keyword specifying there is no
	default \alert{argument/value} for this option. Thus specifying this
	-{}-timing-modes option is compulsory.
	\item [string] Keyword specifying that a string is expected as a value
	to -{}-timing-modes option.\\
	{\small Other possible values are: \alert{number}, \alert{boolean} and two more.}
	\end{itemize}
\end{frame}

\begin{frame}[fragile]{::octopus::extract\_check\_options\_data}{3}
	{\tiny
	\begin{semiverbatim}
	set var_array(10,\alert<5>{timing-modes})	[list "--timing-modes" "<none>" "string" "\alert<1>{1}" "\alert<2>{infinity}" "\alert<3>{T1 T2}" \textbackslash
	
    "\alert<4>{The timing mode(s) the set_case_analysis will be extracted from}" ]
	\end{semiverbatim}
	}
	\begin{itemize}[<+->]
	\item [1] Minimum number of arguments(values).
	\item [infinity] Maximum number of arguments(values).
	\item [T1 T2] Allowed values. Empty string means anything.
	\item [{\small The timi...}] Help associated with the option and displayed when \\
	-{}-help is called.
	\item [{\tiny timing\_modes}] variable will contain all strings/arguments specified after -{}-timing-modes, \alert{timing\_modes} will have a minimum length of 1 and be one of T1 and/or T2.
	\end{itemize}
\end{frame}

\subsection{Messages Handling}
\begin{frame}[fragile]{::octopus::display\_message / ::octopus::summary\_of\_messages}
	\begin{semiverbatim}
	{\small
::octopus::display_message \alert{none} \textbackslash 
    "Extracts DfT constraints from SDC set_case_analysis statements"}
	\end{semiverbatim}
	\begin{itemize}
	\item [none] is message type. \\
	Other types include \alert{error}, \alert{warning}, \alert{workaround}, \alert{info}, \alert{fixme}, \alert{tip}, \alert{debug}.
	\end{itemize}
	\pause	
	\begin{semiverbatim}
		summary_of_messages infos errors
	\end{semiverbatim}
	\begin{itemize}
	\item Displays a user selected list of recorded messages during the flow. Useful at the end of a run.
	\end{itemize}
\end{frame}

\subsection{OctopusRC}
\begin{frame}[fragile]{OctopusRC}{Novel procedures}
	{\small
	\begin{verbatim}
           ::octopusRC::read_dft_abstract_model
           ::octopusRC::define_dft_test_clocks
           ::octopusRC::define_dft_test_signals
           ::octopusRC::constraints_from_tcbs
           
           ::octopusRC::generate_list_of_clock_inverters_for_dft_shell
           ::octopusRC::rec_grouping
           ::octopusRC::report_power_over_area
           \end{verbatim}
	}
\end{frame}
\begin{frame}[fragile]{OctopusRC}{Potential Useful}
	{\small
	\begin{verbatim}      
           ::octopusRC::report_attributes
           ::octopusRC::read_hdl
           ::octopusRC::output_driver
           ::octopusRC::set_attribute_recursive
           
           ::octopusRC::fan_hierarchical
           ::octopusRC::advanced_recursive_grouping
                   
           ::octopusRC::modules_under
	\end{verbatim}
	}
\end{frame}
\begin{frame}[fragile]{OctopusRC}{Put Some Structure}
	{\small
	\begin{verbatim}             
           ::octopusRC::synthesize
           ::octopusRC::read_cpf
           ::octopusRC::write
           ::octopusRC::elaborate
           
           ::octopusRC::set_design_maturity_level
           ::octopusRC::report_timing
	\end{verbatim}
	}
\end{frame}

\subsection{::octopusRC::read\_dft\_abstract\_model}
\begin{frame}[fragile,plain]{::octopusRC::read\_dft\_abstract\_model}
\begin{columns}
\begin{column}{1.1\textwidth}
	\tiny
	\begin{verbatim}
WARNING:   OBSOLETE PROCEDURE ::octopusRC::read_dft_abstract_model:
WARNING:   ::octopusRC::read_dft_abstract_model will dissapear in the future since RC 12.1 is natively supporting this feature.

Usage:
  ::octopusRC::read_dft_abstract_model  --ctl <string>...<string> [--assume-connected-shift-enable] --module <string> --instance <string> [--boundary-opto] [--no-colour] [--debug-level <#>] [--help]

Options:
  --ctl  <string>...<string>       : Specify the CTL file(s) to be read in. If --module option is used then only one file can be read in.
  --assume-connected-shift-enable  : Specify this option if the shift enable is already connected for all CTL files read in. Default value is false.
  --module  <string>               : Specify the module/library cell associated with the ctl file. If missing 'Environment' CTL keyword will be used instead.
  --instance  <string>             : Specify the instance associated with the ctl file. If missing 'Environment' CTL keyword will be used instead.
  --boundary-opto                  : By default, boundary optimization is switched off for modules with CTL associated. By specifying this option you switch boundary optimization on. Default value is false.

General Options:
  --no-colour                      : Turns off colourful output (not recommended). Default value is false.
  --debug-level  <#>               : Displays more debug information during the run. Default value is the calling debug level. Default value is 0.
  --help                           : This help message. Default value is false.

Note:
  --instance option supports only one instance. Thus, the user is strongly encouraged to use native RC commands
  for defining chains on instances. 
\end{verbatim}
  
\end{column}
\end{columns}
\end{frame}

\subsection{::octopusRC::constraints\_from\_tcbs}
\begin{frame}[fragile,plain]{::octopusRC::constraints\_from\_tcbs}
\begin{columns}
\begin{column}{1.1\textwidth}
	\tiny
	\begin{verbatim}
           Extracts the TCB values in a specific mode and writes out constraints

Usage:
  ::octopusRC::constraints_from_tcbs  --tcb-td-file <string>...<string> --mode <string> --exclude-ports <string>...<string> \
                                      --ports <string>...<string> [--no-false-paths] --constraint-file <string> \
                                      [--append] --design <string> [--no-colour] [--debug-level <#>] [--help]

Options:
    --tcb-td-file  <string>...<string>   : TCB test data file(s).
    --mode  <string>                     : TCB mode for which constant values are extracted.
    --exclude-ports  <string>...<string> : Skip the specified TCB port(s) completely.
    --ports  <string>...<string>         : Only this ports are considered. For the rest a false path constraint is added.
    --no-false-paths                     : No false paths are generated for the unconstrained TCB signals. Default value is false.
    --constraint-file  <string>          : The name of the file where the constraints are written into.
    --append                             : Appends into <constraint-file> instead of truncating it. Default value is false.
    --design  <string>                   : Top-Level design.

General Options:
    --no-colour                          : Turns off colourful output (not recommended). Default value is false.
    --debug-level  <#>                   : Displays more debug information during the run. Default value is the calling debug level. Default value is 0.
    --help                               : This help message. Default value is false.

           Note:
           --ports option is compulsory for design maturity higher than bronze, if the design is not in application mode.
                   The reason is that you might hide valid timing paths
	\end{verbatim}
  
\end{column}
\end{columns}
\end{frame}

\subsection{::octopusRC::define\_dft\_test\_signals}
\begin{frame}[fragile,plain]{::octopusRC::define\_dft\_test\_signals}
\begin{columns}
\begin{column}{1.1\textwidth}
	\tiny
	\begin{verbatim}
Usage:
  ::octopusRC::define_dft_test_signals  --timing-modes <string>...<string> [--skip-signals <string>...<string>] \
                                        [--add-signals <string>...<string>] [--test-mode shift|capture] \
                                        [--no-colour] [--debug-level <#>] [--help]

Options:
    --timing-modes  <string>...<string> : The timing mode(s) the set_case_analysis will be extracted from.
    --skip-signals  <string>...<string> : Skip the signal specified in the constraints (not recommended). Default value is false.
    --add-signals  <string>...<string>  : Add more signals then the one specified in the constraints. (not recommended). Default value is false.
    --test-mode  shift|capture          : The DfT mode the timing mode is associated with. Default value is shift.

General Options:
    --no-colour                         : Turns off colourful output (not recommended). Default value is false.
    --debug-level  <#>                  : Displays more debug information during the run. Default value is the calling debug level. Default value is 0.
    --help                              : This help message. Default value is false.

	\end{verbatim}
  
\end{column}
\end{columns}
\end{frame}

\subsection{::octopusRC::define\_dft\_test\_clocks}
\begin{frame}[fragile,plain]{::octopusRC::define\_dft\_test\_clocks}
\begin{columns}
\begin{column}{1.1\textwidth}
	\tiny
	\begin{verbatim}
Usage:
  ::octopusRC::define_dft_test_clocks  --timing-modes <string>...<string> [--skip-clocks <string>...<string>] [--add-clocks <string>...<string>] [--no-colour] [--debug-level <#>] [--help]

Options:
    --timing-modes  <string>...<string> : The timing mode(s) the clocks will be extracted from.
    --skip-clocks  <string>...<string>  : Skip the clocks specified in the constraints. Default value is false.
    --add-clocks  <string>...<string>   : Add more clocks then the one specified in the constraints. Default value is false.

General Options:
    --no-colour                         : Turns off colourful output (not recommended). Default value is false.
    --debug-level  <#>                  : Displays more debug information during the run. Default value is the calling debug level. Default value is 0.
    --help                              : This help message. Default value is false.
	\end{verbatim}
  
\end{column}
\end{columns}
\end{frame}


\section{Contact \& Contribute}
\begin{frame}{Contact \& Contribute}
	\begin{itemize}[<+->]
	\item Octavian Petre
	\item Use git \\
	\alert{git clone https://github.com/octavsly/octopus.git}\\
	Send me the path of your repository to pull your changes from
	\end{itemize}
\end{frame}

\section{Usages}
\begin{frame}{Usages}
	Future presentation about:\\
	\alert{DieHardUS - TCL Utilities and Scripts for your daily work}
	\begin{itemize}
	 \item temposync script using native temposync commands and several type of file lists
	 \item nccoex compilation script using irun.
	 \item tempoflow2TCL file list (alpha stage)
	 \item rtlcompiler scripts (will be discontinued due to Cadence reference flow)
	\end{itemize}
\end{frame}

\begin{frame}{Questions}
\centerline{{\huge ?}}
\end{frame}
\end{document}
           
